
\documentclass[sn-mathphys]{sn-jnl}% Math and Physical Sciences Reference Style
%%\documentclass[sn-aps]{sn-jnl}% American Physical Society (APS) Reference Style
%%\documentclass[sn-vancouver]{sn-jnl}% Vancouver Reference Style
%%\documentclass[sn-apa]{sn-jnl}% APA Reference Style
%%\documentclass[sn-chicago]{sn-jnl}% Chicago-based Humanities Reference Style
%%\documentclass[sn-standardnature]{sn-jnl}% Standard Nature Portfolio Reference Style
%%\documentclass[default]{sn-jnl}% Default
%%\documentclass[default,iicol]{sn-jnl}% Default with double column layout

%%%% Standard Packages
%%<additional latex packages if required can be included here>
%%%%

%%%%%=============================================================================%%%%
%%%%  Remarks: This template is provided to aid authors with the preparation
%%%%  of original research articles intended for submission to journals published 
%%%%  by Springer Nature. The guidance has been prepared in partnership with 
%%%%  production teams to conform to Springer Nature technical requirements. 
%%%%  Editorial and presentation requirements differ among journal portfolios and 
%%%%  research disciplines. You may find sections in this template are irrelevant 
%%%%  to your work and are empowered to omit any such section if allowed by the 
%%%%  journal you intend to submit to. The submission guidelines and policies 
%%%%  of the journal take precedence. A detailed User Manual is available in the 
%%%%  template package for technical guidance.
%%%%%=============================================================================%%%%

\usepackage{aas_macros}

\jyear{2021}%

%% as per the requirement new theorem styles can be included as shown below
\theoremstyle{thmstyleone}%
\newtheorem{theorem}{Theorem}%  meant for continuous numbers
%%\newtheorem{theorem}{Theorem}[section]% meant for sectionwise numbers
%% optional argument [theorem] produces theorem numbering sequence instead of independent numbers for Proposition
\newtheorem{proposition}[theorem]{Proposition}% 
%%\newtheorem{proposition}{Proposition}% to get separate numbers for theorem and proposition etc.

\theoremstyle{thmstyletwo}%
\newtheorem{example}{Example}%
\newtheorem{remark}{Remark}%

\theoremstyle{thmstylethree}%
\newtheorem{definition}{Definition}%

\raggedbottom
%%\unnumbered% uncomment this for unnumbered level heads

\begin{document}

\title{Stellar Bars in Spiral Galaxies Do Not Slow Down}

\maketitle
Elongated bar-like features are ubiquitous, occurring at the centers of
approximately two-thirds of spiral disk galaxies \cite{2000AJ....119..536E,
2007ApJ...657..790M}. Due to gravitational interactions between the bar and
the other components of galaxies, it is expected that angular momentum and
matter will redistribute between galactic components over long (Gyr)
timescales in galaxies hosting a bar \cite{1972MNRAS.157....1L,
1984MNRAS.209..729T, 1985MNRAS.213..451W}. Previous simulation work has
overwhelmingly provided the expectation that the bar pattern will slow its
rotation over time due to a drag caused by material in the dark matter halo on
orbits resonant with the tumbling bar pattern
\cite[e.g.][]{1992ApJ...400...80H, 2000ApJ...543..704D, 2002MNRAS.330...35A,
2002ApJ...569L..83A, 2003MNRAS.341.1179A, 2003MNRAS.346..251O,
2005MNRAS.363..991H, 2006ApJ...637..214M, 2007MNRAS.375..460W,
2009ApJ...697..293D}. Simulations have shown that bars should shed enough
angular momentum to be considered ``slow rotators'' in a few Gyr, but most
observed galaxies seem to be ``fast rotators'' \cite{2011MSAIS..18...23C,
2015AA...576A.102A, 2019MNRAS.482.1733G}. We have performed a simulation of an
isolated galactic disk hosting a strong bar which includes a state-of-the-art
model of the interstellar medium. In this simulation, the bar pattern does not
slow down over time, and instead remains at a stable, constant rate of
rotation. Since gas is torqued by the bar to fall towards the center of the
galaxy, it acts to increase the angular momentum of the bar. However, the
pattern speed we measure is nearly constant over many Gyr, suggesting a novel
equilibrium mechanism is at play - we propose such a mechanism consistent with
our simulation. First, a region of the halo phase space is carved out such
that no resonant drag can occur. If the pattern speed decreases, the
corotation radius becomes larger and so more gas is available for infall since
only gas within corotation can infall \cite[e.g.][]{2011MNRAS.415.1027H}. This
speeds the bar up. If the pattern speed increases, new dark matter which is
resonant with the bar becomes available, slowing the bar down. Thus, the
pattern speed must remain constant. The implications of this are numerous.
First, we resolve a long-standing controversy between the observed fast
rotators and the theoretical expectation that bars should be slow. Second, we
show that the role of gas is of paramount importance in studies which attempt
to uncover the nature of dark matter from its effect of slowing down the bar
\cite{2021MNRAS.500.4710C, 2021MNRAS.505.2412C}. Third, we provide an
explanation for how the Milky Way's bar could be both long-lived and a fast
rotator, of which there is some observational evidence
\cite[e.g.][]{2019MNRAS.490.4740B}. And finally, we complicate the picture of
radial mixing expected to sculpt the Milky Way's disk
\cite[e.g.][]{2012MNRAS.420..913B, 2015ApJ...808..132H}. Our work is a
significant advance in our understanding of the dynamics of barred galaxies.

\backmatter



\bibliography{sn-bibliography}% common bib file
%% if required, the content of .bbl file can be included here once bbl is generated
%%\input sn-article.bbl

%% Default %%
%%\input sn-sample-bib.tex%

\end{document}
