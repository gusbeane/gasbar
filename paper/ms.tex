%\documentstyle[amssymb,12pt,draft,epsf,palatino]{nature-pvd}
\documentclass{natureprintstyle}
\bibliographystyle{naturemag}

\usepackage{aas_macros}

\usepackage{epsfig,caption}
\usepackage{color}
\usepackage{bm}
\usepackage{graphicx}
\usepackage{longtable}
% \usepackage{amssymb}
\usepackage{rotating,xcolor}
\usepackage{hyperref}
% \usepackage{tgbonum}

\usepackage{amsmath}

\usepackage{fontspec}
\setmainfont{texgyrepagella}[
  Extension = .otf,
  UprightFont = *-regular,
  BoldFont = *-bold,
  ItalicFont = *-italic,
  BoldItalicFont = *-bolditalic,
]

\usepackage[misc]{ifsym}
\usepackage{xcolor}

\newcommand{\RCR}{\ensuremath{R_{\textrm{CR}}}}
\newcommand{\Rot}{\ensuremath{\mathcal{R}}}

\title{Stellar Bars in Isolated Gas-Rich Galaxies Do Not Slow Down}

\author{Angus Beane$^{1*}$, Lars Hernquist$^1$, Elena D'Onghia$^{2,3}$, et al.}


\begin{document}

\maketitle

\let\thefootnote\relax\footnote{

\begin{affiliations}
\item Center for Astrophysics $|$ Harvard \& Smithsonian,  Cambridge, MA, USA

\item Department of Physics, University of Wisconsin-Madison, Madison, WI, USA

\item Department of Astronomy, University of Wisconsin-Madison, Madison, WI, USA

$^{*}$ \texttt{\mbox{angus.beane@cfa.harvard.edu}}

\end{affiliations}
}

\vspace{-3.5mm}
\begin{abstract}
  
  Elongated bar-like features are ubiquitous, occuring at the centres of
  approximately two-thirds of spiral disk galaxies \cite{2000AJ....119..536E,
  2007ApJ...657..790M}. Due to gravitational interactions between the bar and
  the other components of galaxies, it is expected that angular momentum and
  matter will redistribute between galactic components over long (Gyr)
  timescales in galaxies hosting a bar \cite{1972MNRAS.157....1L,
  1984MNRAS.209..729T, 1985MNRAS.213..451W}. Previous work has overwhelmingly
  provided the expectation that, due to these interactions, the bar pattern
  will slow its rotation over time \cite{1992ApJ...400...80H,
  2000ApJ...543..704D, 2002MNRAS.330...35A, 2002ApJ...569L..83A,
  2003MNRAS.341.1179A, 2003MNRAS.346..251O, 2005MNRAS.363..991H,
  2006ApJ...637..214M, 2007MNRAS.375..460W, 2009ApJ...697..293D}. We have
  performed a simulation of an isolated galactic disk hosting a strong bar
  which includes a state-of-the-art model of the interstellar medium. Here we
  show that in this simulation the bar pattern does not slow down over time,
  and instead remains at a stable, constant rate of rotation. This behavior
  has been observed in previous simulations but its explanation has remained
  elusive.\cite{1993AA...268...65F, 2010ApJ...719.1470V}. We propose an
  equilibrium mechanism which is consistent with our simulations. This result
  challenges our expectations for how barred galaxies redistribute angular
  momentum and matter over Gyr timescales, which is especially relevant for
  our understanding of how the Milky Way arrived at its present day state.
  
\end{abstract}

\vspace{1cm}

%%%%%%%%%%%%%%%%%%%%%%%%%%%%%%%%%%%%%%%%%%%%%%%%%%%%%

It has long been known that non-axisymmetric features in a stellar disk act to
redistribute mass and angular momentum.\cite{1972MNRAS.157....1L} The
interaction between a bar and a spheroid (either a dark matter halo or stellar
bulge) has received considerable interest.\cite{1984MNRAS.209..729T,
1985MNRAS.213..451W} A reasonable stellar disk in isolation seems to always
generate a bar instability,\cite{1971ApJ...168..343H} but the presence of a
static dark matter halo acts to stabilize the disk against the bar
instability.\cite{1973ApJ...186..467O, 1976AJ.....81...30H} Later, a number of
numerical simulations have confirmed that a live dark matter halo generically
exerts a negative torque on the bar, causing its pattern speed to decrease and
its length and strength to increase\cite{1992ApJ...400...80H,
2000ApJ...543..704D, 2002MNRAS.330...35A, 2002ApJ...569L..83A,
2003MNRAS.341.1179A, 2003MNRAS.346..251O, 2005MNRAS.363..991H,
2006ApJ...637..214M, 2007MNRAS.375..460W, 2009ApJ...697..293D}. Intuitively,
the bar excites a wake of resonant material in the dark matter halo in a
manner similar to dynamical friction. This wake lags and thus exerts a
negative torque on the bar.

The pre-eminent expectation that the dark matter halo acts to slow the bar
down conflicts with observational estimates of bar pattern speeds. Bar
rotation rates are typically classified by the dimensionless ratio
\begin{equation}
\Rot = \RCR/R_b\textrm{,}
\end{equation}
where \RCR{} is the radius of corotation and $R_b$ is the length of the
bar.\footnote{The radius of corotation \RCR{} is defined for circular orbits as
the radius at which the orbital frequency is equal to the pattern speed,
$\Omega_p$, of a given non-axisymmetric feature. In a galaxy with a constant
circular velocity $V_c$, it is given by $\RCR = V_c / \Omega_p$.} Galaxies
with $\Rot < 1.4$ are considered ``fast rotators'' while galaxies with $\Rot >
1.4$ are considered ``slow rotators.''\cite{2000ApJ...543..704D} Galaxies with
$\Rot < 1$ are not thought to be stable.\cite{1980AA....81..198C}
Observational estimates of the pattern speeds of bars indicate that nearly all
galaxies have $1 < \Rot < 1.4$.\cite{2011MSAIS..18...23C, 2015AA...576A.102A,
2019MNRAS.482.1733G} 

The role of gas on the evolution of the bar is less well-understood. Since the
gas phase typically only contributes about $10-20\%$ of the mass of a galaxy
at the present day, one might naively expect it to have a subdominant effect
on the bar. However, because gas is collisional, it can participate in
non-resonant angular momentum exchange with the bar. Thus, numerical work has
shown that the gas phase has an outsized influence on the
bar.\cite{2010ApJ...719.1470V, 2013MNRAS.429.1949A}

We have performed a simulation of a disk galaxy using the finite-volume
gravito-hydrodynamics code AREPO.\cite{2010MNRAS.401..791S} In AREPO, the
fluid is discretized as a moving Voronoi mesh. We use the additional physics
in the galaxy formation module Stars and MUltiphase Gas in GaLaxiEs (SMUGGLE)
\cite{2019MNRAS.489.4233M}. SMUGGLE is a comprehensive and self-consistent
galaxy formation model with a wide range of physical processes, including
radiative heating/cooling, star formation, and stellar feedback. More detailed
information on SMUGGLE is given in the Methods section. Our disk galaxy is a
modified version of the GALAKOS model\cite{2020ApJ...890..117D}, which
consists of a stellar disk, stellar bulge, and dark matter halo. After
$\sim2.5\,\textrm{Gyr}$ of evolution, the GALAKOS disk forms a bar consistent
with the Milky Way bar in terms of pattern speed and length
($\sim40\,\textrm{km}/\textrm{s}/\textrm{kpc}$ and $\sim4.5\,\textrm{kpc}$,
respectively). We modify this setup to include a gas phase, the details of
which is given in the Methods section.

A surface density projection of our simulatons is shown in
Fig.~\ref{fig:overview}. The upper three panels show the disk in the N-body
run while the lower three panels show the disk in the SMUGGLE run. Each column
shows the disk $\sim1\,\textrm{Gyr}$ apart in time. There is a large qualitative
difference in the evolution of the bar pattern between the two runs. We see
that in the N-body case, the bar lengthens in time and grows in strength. In
the SMUGGLE case, the bar pattern retains a similar length and strength
between panels.

We show the time evolution of different bar properties in Fig.~\ref{fig:prop}.
In the upper panel, we show the pattern speed over time in the N-body (blue)
and SMUGGLE (orange) runs. The pattern speed in the N-body case slows down
while the pattern speed in the SMUGGLE case remains constant. The slowing down
of the pattern speed in the N-body case is consistent with a long line of
numerical research on bars in N-body simulations\cite{1992ApJ...400...80H,
2000ApJ...543..704D, 2002MNRAS.330...35A, 2002ApJ...569L..83A,
2003MNRAS.341.1179A, 2003MNRAS.346..251O, 2005MNRAS.363..991H,
2006ApJ...637..214M, 2007MNRAS.375..460W, 2009ApJ...697..293D}. However, in
the SMUGGLE case the pattern speed remains constant. After the first Gyr of
evolution, we find that the pattern speed increases by only $\sim10\%$ over
the next $4\,\textrm{Gyr}$, compared to a $\sim43\%$ decrease in the pattern
speed for the N-body run over the same interval. As we saw
qualitatively in Fig.~\ref{fig:overview}, the length of the bar in the N-body
case grows over time while it remains constant in the SMUGGLE case.

The bottom panel of Fig.~\ref{fig:prop} shows the torque exerted on the bar by
different components. The solid lines indicate the torque on the bar by the
dark matter halo whereas the dashed line indicates the torque on the bar by
the gas phase. In the N-body case, the halo exerts a steady negative torque on
the bar, with an average torque from $1$ to $4\,\textrm{Gyr}$ of $-58.0$ in
units of $10^{10}M_{\odot}\,(\textrm{km}/\textrm{s})^2$. The halo in the
SMUGGLE case exerts a similar negative torque on the bar in the first Gyr of
evolution, but after that the halo exerts a much smaller torque on the bar,
averaging only $-7.8$ in the same units and over the same interval. The gas in
the SMUGGLE case exerts a steady positive torque averaging $11.7$ in the same
units and over the same interval.

The fact that the dark matter halo in the SMUGGLE case exerts a smaller
positive torque on the bar can be understood in terms of the halo wake
mechanism. In the N-body case, halo material which is resonant with the bar
will form a wake which lags behind and exerts a negative torque on the bar,
which slows it down.\cite{1984MNRAS.209..729T, 1985MNRAS.213..451W,
1992ApJ...400...80H}\footnote{Since the bar is not a solid body, it is not
guaranteed that a negative torque will slow it down - e.g. a negative torque
could shred the bar, reducing its moment of inertia without changing its
pattern speed. However, the bar seems to empirically respond to a negative
torque induced by a halo wake by slowing down.} As the bar slows down, the
location of the resonances in phase space changes, allowing halo material
newly resonant with the bar to form a new wake. However, the gas is a reliable
source of positive torque on the bar, speeding the bar up. This stops the
resonance location from changing such that the halo cannot form a new wake,
arresting the process by which the halo can slow the bar down.

We can test this interpration by measuring the angle offset between the halo
wake and the bar. If the wake and the bar are aligned (i.e., there is no angle
offset), then the wake cannot exert a negative torque on the bar. This angle
is plotted in the middle panel of Fig.~\ref{fig:wake}, which shows that the
angle offset is larger in the N-body case than in the SMUGGLE case by about a
factor of two. The left and right panels of Fig.~\ref{fig:wake} show the halo
wake with respect to the location of the bar in the N-body (left) and SMUGGLE
(right) cases.

The presence of the gas can arrest the process by which the dark matter halo
wake forms. However, this does not explain why the pattern speed in the
SMUGGLE case is nearly constant over several Gyr. Naively, it would be a
coincidence that the bar pattern speed remains constant in the SMUGGLE case,
resulting from a chance cancellation of the halo and gas torques. However, a
constant pattern speed in the presence of gas has been observed in a few
simulations of barred galaxies with gas.\cite{1993AA...268...65F,
2010ApJ...719.1470V} Previous work has argued this is due to the bar torquing
gas inwards, but no explanation has been given for why it might remain
constant.

We propose that an equilibrium mechanism is responsible for the pattern speed
remaining approximately constant. In this scenario, a torque must oppose
changes in the pattern speed so that when the bar speeds up, a negative torque
will slow it down and when the bar slows down, a positive torque will speed it
up. It is simple to explain the first of these - when the pattern speed
increases, the location of resonances will shift to regions of the dark matter
halo phase space where no wake has been excited yet (e.g., the corotation
radius will shrink).

When the bar slows down, we argue that this induces a larger positive torque
from the gas phase. Only gas within corotation will flow inwards, while gas
outside corotation will flow outwards.\cite{2011MNRAS.415.1027H} Since the
corotation radius is larger for more slowly rotating bars, it follows that
more slowly rotating bars should be more efficient at driving gas inflows and
thus experience a larger positive torque from the gas phase. We performed an
experiment to test this hypothesis by forcing the stellar disk in the SMUGGLE
run to rotate at a constant angular rate and measuring the torque on the bar
by the gas phase at different rotation rates. The result of this experiment is
shown in Fig.~\ref{fig:equil}, which shows that a more slowly rotating bar
experiences a larger positive torque from the gas.

We have therefore shown evidence for an equilibrium mechanism at play which
keeps the pattern speed of the bar constant, resulting from the complex
interplay between the dark matter halo and the gas phase.

The implications of this finding are numerous. First, we naturally explain why
nearly all observed galaxies are fast rotators without requiring the inner
regions of dark matter halos to be underdense\cite{1998ApJ...493L...5D,
2000ApJ...543..704D} or introducing new physics.\cite{2021MNRAS.503.2833R,
2021MNRAS.508..926R} Second, we show that the role of gas is of paramount
importance in studies which attempt to uncover the nature of dark matter from
its effect of slowing down the bar.\cite{2021MNRAS.500.4710C,
2021MNRAS.505.2412C} Third, we provide an explanation for how the Milky Way's
bar could be both long-lived and a fast rotator, of which there is some
observational evidence.\cite{2019MNRAS.490.4740B} And finally, we complicate
the picture of radial mixing expected to sculpt the Milky Way's disk
\cite{2012MNRAS.420..913B, 2015ApJ...808..132H}, a process which relies upon
the pattern speed of the bar to change with time (though our work does not
alter expectations for radial mixing induced by spiral arms\cite{2002MNRAS.336..785S}).

Barred galaxies in cosmological simulations of galaxy formation continue to be
in conflict with observations.\cite{2017MNRAS.469.1054A, 2019MNRAS.483.2721P,
2021AA...650L..16F} However, the pattern speeds of bars in both cosmological
simulations and the real universe can be affected by environmental processes
not included in our simulation -- e.g., satellite
infall\cite{2011Natur.477..301P}, non-sphericity\cite{2013MNRAS.429.1949A} or
rotation\cite{2013MNRAS.434.1287S, 2014ApJ...783L..18L, 2018MNRAS.476.1331C,
2019MNRAS.488.5788C} in the dark matter halo, or perhaps even the gaseous
circumgalactic medium. Naturally, extending our present work to account for
such effects is a crucial next step in understanding the formation and
evolution of galactic bars.

\begin{figure*}[h]%
\centering
\includegraphics[width=0.8\textwidth]{fig/fig1.pdf}
\caption{Surface density projections in an N-body only simulation (upper
panels) and a simulation which includes the SMUGGLE model (bottom panels). We
can see that in the N-body run, the bar grows in length and strength. In the
SMUGGLE run (which includes a gaseous phase and a model for the multi-phase
interstellar medium), the bar remains at approximately the same length and
strength over the course of the simulation. The N-body model is identical to
the GALAKOS model, discussed in the text.}\label{fig:overview}
\end{figure*}

\begin{figure}[h]%
\centering
\includegraphics[width=0.45\textwidth]{fig/fig2.pdf}
\caption{Various bar properties from the N-body and SMUGGLE runs. The
\textit{upper panel} show the evolution of the pattern speed. As expected, the
bar in the N-body run slows down due to interactions between the bar and the
dark matter halo. However, the bar in the SMUGGLE run does not slow down and
instead remains at a constant pattern speed. The \textit{middle panel} shows
the evolution of the bar length. In the N-body case, the bar lengthens. This
occurs because as the pattern speed drops, bar-like orbits at larger radii are
possible. Stars are captured on these orbits, lengthening the bar. This
process does not occur in the SMUGGLE cases since the bar pattern speed is not
decreasing, and therefore the bar length remains constant. The \textit{lower
panel} shows the torque on the bar by different components. The solid lines
show the torque exerted by the halo in both the N-body and SMUGGLE cases. The
dashed line shows the torque exerted by the gas phase in the SMUGGLE run
(there is no gas in the N-body run). The first $1.5\,\textrm{Gyr}$ of
evolution of the N-body model is not shown. Details on how these properties
are calculated is given in the Methods section.}\label{fig:prop}
\end{figure}

\begin{figure*}[h]%
\centering
\includegraphics[width=0.9\textwidth]{fig/fig3.pdf}
\caption{The wake excited in the dark matter halo in the N-body case
(\textit{left panel}) and SMUGGLE case (\textit{right panel}) after
$2.6\,\textrm{Gyr}$. The \textit{left} and \textit{right panels} show a
surface density projection in the $x$-$y$ plane of the dark matter halo after
an axisymmetric average has been subracted. The solid line indicates the
direction of the bar while the dashed line indicates the direction of the halo
wake (both measured by taking the second Fourier component within a sphere of
all material within a radius of $X\,\textrm{kpc}$). The \textit{center panel}
shows the time evolution of the angle difference between the bar and the halo
wake, as measured from the second Fourier component. After the first Gyr, the
angle difference in the SMUGGLE case is smaller than in the N-body case by
about a factor of two, reflecting how the dark matter halo in the SMUGGLE case
is unable to exert as negative a torque on the bar as in the N-body
case.}\label{fig:wake}
\end{figure*}

\begin{figure}[h]
\centering
\includegraphics{fig/fig4.pdf}
\caption{The torque on the bar by the gas when the pattern speed of the bar is
kept at a constant value. Only gas within the corotation radius is able to
infall. Since slower bars have larger corotation radii, slower bars experience
a larger net torque than faster bars. The setup of the simulations used here
is identical to the SMUGGLE case discussed earlier and in the Methods section,
except the N-body disk is rotated as a solid body with a constant angular
velocity.}\label{fig:equil}
\end{figure}


%%%%%%%%%%%%%%%%%%%%%%%%%%%%%%%%%%%%%%%%%%%%%%%%%%%%

\bibliography{ref}

\begin{addendum}
  
\item [Acknowledgements] Foo.

\item[Author Contributions] Foo.

  \item[Data Availability] Foo.
    
  \item[Code Availability] Foo.
    
\end{addendum}


%%%%%%%%%%%%%%%%%%%%%%%%%%%%%%%%%%%%%%%%%%%%%%%%%%%%%%%%%%
%%%%%%%%%%%%%%%%%%%%%%%%%%%%%%%%%%%%%%%%%%%%%%%%%%%%%%%%%%

\newpage

\setcounter{page}{1}
\setcounter{figure}{0}
\setcounter{table}{0}
\captionsetup[figure]{labelformat=empty}
\renewcommand{\figurename}{Extended Data Figure}
\renewcommand{\thetable}{Extended Data \arabic{table}}

\begin{center}
{\bf \Large \uppercase{Methods} }
\end{center}

\noindent
{\bf SMUGGLE Model}
\\
\noindent
We use the Stars and MUltiphase Gas in GaLaxiEs (SMUGGLE) model
\cite{2019MNRAS.489.4233M} implemented within the moving-mesh, finite-volume
hydrodynamics code AREPO \cite{2010MNRAS.401..791S}. The SMUGGLE model
includes self-gravity, hydrodynamics, radiative heating and cooling, star
formation, and stellar feedback. Explicit gas cooling and heating of the
multi-phase interstellar medium is implemented, covering temperature ranges
between $10$ and $10^8\,\textrm{K}$.

Star formation occurs in cells above a density threshold
($n_{\textrm{th}}=100\,\textrm{cm}^{-3}$) according to SH03 with a
star-formation efficiency of $\epsilon = 0.01$. Star formation converts gas
cells into star particles which represent single stellar populations. For each
star particle, the deposition of energy, momentum, and mass from stellar winds
and supernovae is modeled. Photoionization and radiation pressure are modeled
using an approximate treatment. A more detailed description of this model can
be found in the flagship SMUGGLE paper.\cite{2019MNRAS.489.4233M}

We used the fiducial model parameters, except that we increased the number of
effective neighbors $N_{\textrm{ngb}}$ for the deposition of feedback from
$64$ to $512$. We found that a lower value of $N_{\textrm{ngb}}$ resulted in
inefficient photoionization feedback since the photoionizing budget had not
been exhausted after deposition into $64$ neighboring cells. We also used an
updated version of SMUGGLE using a new mechanical feedback routine similar to
the one described in ref.\cite{2018MNRAS.480..800H} This updated routine is a
tensor renormalization which ensures linear and angular momentum conservation
to machine precision.

\vspace{12pt}

\noindent
{\bf Initial Setup}
\\
\noindent
The initial setup of the galactic disk used in this work follows closely the
GALAKOS model\cite{2020ApJ...890..117D}, which uses a modified version of
\texttt{MakeNewDisk}.\cite{2005MNRAS.361..776S} The GALAKOS model has three
components - a radially exponential and vertically isothermal stellar disk,
and a stellar bulge and dark matter halo following a Hernquist
profile.\cite{1990ApJ...356..359H} All N-body runs in this work used the same
setup parameters as the GALAKOS disk, more details of which can be found in
the original paper.

The addition of the gas phase was done as follows. The version of
\texttt{MakeNewDisk} used for the original GALAKOS model can generate a gas
disk which is radially exponential and in vertical gravito-hydrodynamic
balance. We modified the radial profile of this code in order to allow us to
generate a disk with a constant surface density within some cut-off radius,
and then exponentially declining beyond that radius with the scale-length of
the stellar disk. We used an initial surface density of
$20\,M_{\odot}/\textrm{pc}^2$ and a cut-off radius of $9.3\,\textrm{kpc}$.

After generating the gaseous disk in this way, we stitched the gas disk
together with the GALAKOS N-body disk (and bulge and dark matter halo) after
the GALAKOS disk has been allowed to evolve for $1.5\,\textrm{Gyr}$. The
purpose of allowing the GALAKOS disk to evolve first for a short period of
time is to allow for the bar to form unimpacted by the presence of the gas
(which would normally disrupt the formation of the bar). We made one
additional modification when stitching the gas disk together with the N-body
disk - we created a vacuum within the central $4\,\textrm{kpc}$. This vacuum
guards against an initial dramatic infall of gas within the bar region, which
we found to destroy the bar.

Our setup is initially out of equilibrium, but we found that after
$\sim500\,\textrm{Myr}$, the entire system has settled into a steady-state
configuration and initial transients appear not to affect the results after
this point. The constant surface density of the initial gas disk is important
for ensuring the gas disk is dense enough in order for comparisons to real
galaxies to be appropriate. We plot the circular velocity curve in Extended
Data Fig.~\ref{fig:vcirc} compared to observational
estimates.\cite{2019ApJ...871..120E} Our disk is slightly more massive than
the Milky Way disk, and in the SMUGGLE run the addition of the gas phase
results in a slightly higher circular velocity. We also show the evolution of
the surface density profile in Extended Data Fig.~\ref{fig:surf} compared to
observational estimates (CITE).

We used a mass resolution of $7.5\times10^3\,M_{\odot}$ for the baryonic
components (initial stellar disk, stellar bulge, and gas) and a mass
resolution of $3.75\times10^4\,M_{\odot}$ for the dark matter halo. This mass
resolution is closest to ``level 3'' in the AURIGA
simulations.\cite{2017MNRAS.467..179G} This corresponds to approximately
$6.4\times10^6$ particles in the stellar disk, $1.1\times10^6$ in the bulge,
$1.2\times10^6$ in the gas disk, and $25.3\times10^6$ in the dark matter halo.
We used a softening length of $0.02\,\textrm{kpc}$ for all components.

\begin{figure}[h]%
\centering
\includegraphics[width=0.45\textwidth]{fig/fig-vcirc.pdf}
\caption{The circular velocity curve for the N-body run (blue) and the SMUGGLE
run (orange) compared to observational estimates.\cite{2019ApJ...871..120E} We
see that the circular velocity curve for both runs is marginally larger than
the Milky Way's, but still comparable. The SMUGGLE circular velocity curve is
larger than the N-body curve due to the additional mass in the gas phase.}
\label{fig:surf}
\end{figure}

\begin{figure}[h]%
\centering
\includegraphics[width=0.45\textwidth]{fig/fig-surf.pdf}
\caption{The time evolution of the gas surface density (\textit{upper}) and the star formation rate (SFR) surface density (\textit{lower}).  } \label{fig:surf}
\end{figure}

\vspace{12pt}

\noindent
{\bf Bar analysis}
\\
\noindent
The analysis of various bar properties is performed as follows. First, the
pattern speed is measured from the angle of the second Fourier component. We measured the second Fourier component by computing,
\begin{equation}
\begin{split}
A_2 &= \sum_i m_i e^{i 2 \phi_i} \\
A_0 &= \sum_i m_i \textrm{,}
\end{split}
\end{equation}
where $m_i$ and $\phi_i$ are the mass and azimuthal angle of each particle, respectively. We
computed $A_2$ and $A_0$ in cylindrical bins of width $0.5\,\textrm{kpc}$ from radii of
$0$ to $30\,\textrm{kpc}$. We defined the angle of the bar $\phi_b$ to be
twice the angle of complex number $A_2$ as measured in the bin extending from
a radius of $2.5$ to $3\,\textrm{kpc}$. After correcting for the periodicity
of $\phi_b$, we measured the pattern speed as the two-sided finite gradient of
$\phi_b$ as a function of time.

The time evolution of the bar strength, defined as the maximum of
$\left|A_2/A_0\right|$ as a function of radius, is shown in Extended Data
Fig.~\ref{fig:strength}. The quantity $\left|A_2/A_0\right|$ varies from $0$
to $1$, with larger values indicating a stronger bar pattern. We see that in
the N-body case, $\left|A_2/A_0\right|$ increases over time as the bar pattern
slows. This is consistent with previous N-body simulations which showed a
clear correlation between the bar pattern speed and the bar
strength.\cite{2003MNRAS.341.1179A} In the SMUGGLE case, we see that the bar
strength has an initial drop but then remains at a roughly constant but
slightly decreasing strength. This is consistent with the pattern speed in the
SMUGGLE case being roughly constant or slightly increasing.

\begin{figure}[h]%
\centering
\includegraphics[width=0.45\textwidth]{fig/fig-A2.pdf}
\caption{The time evolution of the bar strength, measured as the second
Fourier component divided by the zeroth Fourier component (a formula is given
in the Methods section). We see that in the N-body case (blue) the bar
strength increases with time, consistent with previous results showing that
the bar strength increases as bars slow down. In the SMUGGLE case (orange), we
see that the bar strength remains roughly constant, possibly slightly
decreasing with time. This is also consistent with the expected relation
between pattern speed and strength since the bar in this case is not slowing
down.}
\label{fig:strength}
\end{figure}

Computing the length of the bar and the torque on the bar by different
components requires us to decompose the disk into a component which is trapped
by the bar and a component which is untrapped. In order to do this, we follow
closely the technique developed in ref.\cite{2016MNRAS.463.1952P} We analyzed
the orbit of each star particle (meaning initial disk, bulge, and newly formed
stars) by extracting the $x$-$y$ positions of the apoapse of each in a frame
corotating with the bar (apoapses are defined as local maxima in $r$). For
each apoapse, we searched for the $19$ closest apoapses in time and applied a
$k$-means clustering algorithm on this set of $20$ points with $k=2$. We then
computed for each of the two clusters the average angle from the bar
$\left<\Delta \phi\right>_{0,1}$, the standard deviation in $R$ of the points
${\sigma_R}_{0,1}$, and the average radius of the cluster
$\left<R\right>_{0,1}$. At each apoapse, a particle was considered to be in
the bar if it met the following criteria:
\begin{equation}
\textrm{max}\left(\left<\Delta \phi\right>_{0,1}\right) < \pi / 8
\end{equation}
\begin{equation}
\frac{{\sigma_R}_0 + {\sigma_R}_1}{\left<R\right>_0 + \left<R\right>_1} < 0.22
\end{equation}
These criterion are slightly different and simplified from the ones used in
ref.\cite{2016MNRAS.463.1952P}, but we found to empirically work well at
decomposing the disk into a bar and disk component. In Extended Data
Fig.~\ref{fig:decomp}, we show an example of this decomposition. The
\textit{left} panel shows a surface density projection of the stellar disk and
bulge (including newly formed stars) from the SMUGGLE model after
$1\,\text{Gyr}$ of evolution in a frame such that the bar is aligned with the
$x$-axis. The \textit{middle} panel shows a projection of the subset of stars
that are identified as being trapped in the bar and the \textit{right} panel
shows a projection of the stars that are not identified as being trapped. The
fact that the \textit{right} panel is roughly axisymmetric indicates the bar
decomposition is performing adequately.

\begin{figure*}[h]%
\centering
\includegraphics[width=0.95\textwidth]{fig/fig-bar_decomp.pdf}
\caption{Demonstration of our bar decomposition procedure based on
ref.\cite{2016MNRAS.463.1952P} The \textit{left panel} shows a surface density
projection through the stellar component of the N-body simulation (disk and
bulge). The \textit{middle panel} shows the component of the disk identified
as being trapped in the bar while the \textit{right panel} shows the component
of the disk identified as not being trapped in the bar. The fact that the
untrapped stars form a roughly axisymmetric structure indicates our bar
decomposition is sufficiently accurate. {\color{red} I will remake this figure
- I ended up losing the files which stored the bar membership, so I just
pulled these screenshots from an earlier presentation.}}
\label{fig:decomp}
\end{figure*}

After the disk has been decomposed into a trapped and untrapped component, we
measured the bar length as being the radius $R_b$ which encapsulates $99\%$ of
the stars identified as being trapped in the bar, allowing for some outliers.
For the computation of the torque, we used the tree algorithm in
\texttt{MakeNewDisk} customized to be accessible from \texttt{Python} using
\texttt{Cython}. This algorithm is based on the \texttt{TREESPH} code. We
constructed a tree using only the star particles identified as being trapped
in the bar. We then queried the tree at the locations of all resolution
elements in the other components and computed the torque of the bar on such
components. The torque on the bar by the other components is simply the
negative of the torque on the other components by the bar. A similar method
was applied in measuring the torque for the disk when the pattern speed is
kept constant, as is done in Fig.~\ref{fig:equil}.

\vspace{12pt}

\noindent
{\bf Title}
\\
\noindent
Next section.

% \bibliography{ref2}



\end{document}



