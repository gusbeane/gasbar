% mnras_template.tex
%
% LaTeX template for creating an MNRAS paper
%
% v3.0 released 14 May 2015
% (version numbers match those of mnras.cls)
%
% Copyright (C) Royal Astronomical Society 2015
% Authors:
% Keith T. Smith (Royal Astronomical Society)

% Change log
%
% v3.0 May 2015
%    Renamed to match the new package name
%    Version number matches mnras.cls
%    A few minor tweaks to wording
% v1.0 September 2013
%    Beta testing only - never publicly released
%    First version: a simple (ish) template for creating an MNRAS paper

%%%%%%%%%%%%%%%%%%%%%%%%%%%%%%%%%%%%%%%%%%%%%%%%%%
% Basic setup. Most papers should leave these options alone.
\documentclass[a4paper,fleqn,usenatbib]{mnras}

% MNRAS is set in Times font. If you don't have this installed (most LaTeX
% installations will be fine) or prefer the old Computer Modern fonts, comment
% out the following line
\usepackage{newtxtext,newtxmath}
% Depending on your LaTeX fonts installation, you might get better results with one of these:
%\usepackage{mathptmx}
%\usepackage{txfonts}

% Use vector fonts, so it zooms properly in on-screen viewing software
% Don't change these lines unless you know what you are doing
\usepackage[T1]{fontenc}
\usepackage{ae,aecompl}


%%%%% AUTHORS - PLACE YOUR OWN PACKAGES HERE %%%%%

% Only include extra packages if you really need them. Common packages are:
\usepackage{graphicx}	% Including figure files
\usepackage{amsmath}	% Advanced maths commands
\usepackage{amssymb}	% Extra maths symbols

%%%%%%%%%%%%%%%%%%%%%%%%%%%%%%%%%%%%%%%%%%%%%%%%%%

%%%%% AUTHORS - PLACE YOUR OWN COMMANDS HERE %%%%%
\newcommand{\pc}{\ensuremath{\text{pc}}}
\newcommand{\kpc}{\ensuremath{\text{kpc}}}

\newcommand{\Msun}{\ensuremath{\text{M}_{\odot}}}
\newcommand{\K}{\ensuremath{\text{K}}}

\newcommand{\beq}{\begin{equation}}
\newcommand{\eeq}{\end{equation}}

\newcommand{\mwib}{\texttt{mwib}}

\newcommand{\arepo}{\textsc{Arepo}}
\newcommand{\areport}{\textsc{Arepo-RT}}
\newcommand{\smuggle}{\textsc{SMUGGLE}}

% Please keep new commands to a minimum, and use \newcommand not \def to avoid
% overwriting existing commands. Example:
%\newcommand{\pcm}{\,cm$^{-2}$}	% per cm-squared

%%%%%%%%%%%%%%%%%%%%%%%%%%%%%%%%%%%%%%%%%%%%%%%%%%

%%%%%%%%%%%%%%%%%%% TITLE PAGE %%%%%%%%%%%%%%%%%%%

% Title of the paper, and the short title which is used in the headers.
% Keep the title short and informative.
\title[Star Bar]{On Star Formation at Dynamical Resonances of the Bar}

% The list of authors, and the short list which is used in the headers.
% If you need two or more lines of authors, add an extra line using \newauthor
\author[A. Beane et al.]{
% Angus Beane,$^{1}$\thanks{E-mail: abeane@cfa.harvard.edu}
Angus~Beane$^{1}$,\thanks{E-mail: angus.beane@cfa.harvard.edu}
Elena~D'Onghia$^{2,3}$,
Lars~Hernquist$^{1}$,
and Federico~Marinacci$^{4}$
\\
% List of institutions
% Center for Astrophysics {\normalfont |} Harvard \& Smithsonian, 60 Garden Street, Cambridge, MA 02138, USA
$^{1}$Center for Astrophysics {\normalfont |} Harvard \& Smithsonian, 60 Garden Street, Cambridge, MA 02138, USA\\
$^{2}$Department of Astronomy, University of Wisconsin, 475 North Charter Street, Madison, WI 53706, USA\\
$^{3}$Center for Computational Astrophysics, Flatiron Institute, 162 5th Avenue, New York, NY 10010, USA\\
$^{4}$Department of Physics \& Astronomy, University of Bologna, via Gobetti 93/2, 40129 Bologna, Italy
% $^{3}$Department of Physics \& Astronomy, University of Pennsylvania, 209 South 33rd Street, Philadelphia, PA 19104, USA
}

% These dates will be filled out by the publisher
\date{Accepted XXX. Received YYY; in original form ZZZ}

% Enter the current year, for the copyright statements etc.
\pubyear{2019}

% Don't change these lines
\begin{document}
\label{firstpage}
\pagerange{\pageref{firstpage}--\pageref{lastpage}}
\maketitle

% Abstract of the paper
\begin{abstract}
Long-term, secular processes can be important drivers of present-day galactic
properties. The Milky Way is known to host one and perhaps multiple bars whose
gravitational influence has observable consequences on both stellar and
gaseous components. It has long been known that along the major axis of a bar
gas can be driven towards the galactic center, fueling, e.g., the Central
Molecular Zone and nuclear activity. On the other hand, stars are known to
congregate near Lagrange points (i.e., minimums in the galactic potential),
forming known moving groups such as Hercules. We report the results of
isolated simulations incorporating both a state-of-the-art star formation and
feedback model (in which a multi-phase interstellar medium is generated
self-consistently) and a model of the Milky Way which forms a strong bar with
a pattern speed matched to current observations. We are able to show, for the
first time, that an enhancement of star formation may occur at the Lagrange
points. We derive a collisional analogue of the theory of dynamical resonances
and
\end{abstract}

% Select between one and six entries from the list of approved keywords.
% Don't make up new ones.
\begin{keywords}
Galaxy: disc -- Galaxy: kinematics and dynamics -- stars: kinematics and dynamics
\end{keywords}

%%%%%%%%%%%%%%%%%%%%%%%%%%%%%%%%%%%%%%%%%%%%%%%%%%

%%%%%%%%%%%%%%%%% BODY OF PAPER %%%%%%%%%%%%%%%%%%

\section{Introduction}
Roughly two-thirds of spiral galaxies are known to host bars in the
near-infrared, about half of which are strong
\citep[e.g.][]{2000AJ....119..536E}. Given their ubiquity, it is no surprise
that the Milky Way is known to host a strong bar. A full understanding of the
origin of the Galaxy's structure clearly rests on the impact of the bar not
just in the inner region but also near the Solar circle. Its impact on the gas
phase is highly discussed in the inner regions of the Galaxy, but studies of
the bar's impact on star formation at the Solar circle are less developed. In
this work, we seek to contribute to our understanding of the latter topic.



\section{Methods}
\subsection{Numerical Methods and Physical Model}
We perform our simulations using the moving-mesh code \arepo{}. Gravity is
solved using a standard oct-tree algorithm \citep{1986Natur.324..446B}.
Hydrodynamics is modelled with a finite volume solver over an unstructured
Voronoi mesh allowed to move in a quasi-Lagrangian fashion
\citep{2010MNRAS.401..791S, 2016MNRAS.455.1134P}. A concise summary of this
core functionality of \arepo{} is available with its public
release\footnote{\url{https://arepo-code.org}} \citep{2019arXiv190904667W}.

In our simulations we will reach $\sim10^3\,\Msun$ mass resolution. As a
result, the cold phase of the interstellar medium (ISM) is well-resolved
(\textcolor{red}{cite}), and an appropriate physical model which account for
the processes occuring at these scales must be used. In this vein, we use the
state-of-the-art model \smuggle{} \citep{2019MNRAS.489.4233M}. A detailed
description of the model is presented in \citet{2019MNRAS.489.4233M}, but we
summarize the salient components here.

Standard cooling processes are accounted for using tabulated \textsc{cloudy}
calculations, described in detail in \citet{2013MNRAS.436.3031V}. In addition
to standard channels, metal line, fine-structure, and molecular cooling
processes are also accounted for, allowing the gas to reach
$\sim10\,\text{K}$, as well as self-shielding for high-density gas. Cosmic ray
and photoelectric heating are also taken into account. Star formation follows
a standard probabilistic approach \citep{2003MNRAS.339..289S}, with the star
formation efficiency ($\epsilon$) set to $0.01$ and additional density and
viriality conditions. Finally, stellar feedback is accounted for in three
different channels: supernovae, young massive stellar radiation, and stellar
winds from OB and AGB stars.

The model used here is similar to the one used first by
\citet{2011MNRAS.410.1391A, 2013ApJ...770...25A} and by the Feedback In
Realistic Environments (FIRE) collaboration \citep{2011MNRAS.417..950H,
2014MNRAS.445..581H, 2018MNRAS.480..800H}, with some key differences discussed
in \citet{2019MNRAS.489.4233M}. One drawback of the \smuggle{} method is that
radiation is not self-consistently tracked and evolved, and therefore
radiation feedback must be implemented in an approximate manner. An extension
of \smuggle{} which explicitly tracks the radiation field as an additional
fluid component \citep{2019arXiv191014041K} based upon \areport{}
\citep{2019MNRAS.485..117K} is available. We plan to use this module in the
future.

\subsection{Initial Conditions}\label{ssec:ics}
The starting point for the construction of our initial conditions is the
N-body model GALAKOS \citep{2020ApJ...890..117D}. GALAKOS has structural
parameters and a circular velocity curve which are in reasonable agreement
with current data. However, GALAKOS is a particularly good starting point
because as \citet{2020ApJ...890..117D} showed, the disk forms a strong bar
after ~ 1 Gyr. The bar then grows, until at ~ 2.5 Gyr it has grown and slowed
down to a pattern speed of ~40 km/s/kpc. Evidence is mounting that the Milky
Way's bar is long and slow, just like that in GALAKOS.

We convert a fraction (10\%) of the initial stellar disk of GALAKOS to a gas
disk. As described in \citet{2005MNRAS.361..776S}, the gas disk is initially
in hydrostatic equilibrium. We assumed an ideal EOS for the gas phase,
consistent with our underlying physical model. We make two further
modifications to this setup. First, we ``core'' the gas profile -- instead of
assuming the gas surface density profile is exponential, we assume it is a
constant value within some radius $R_{\text{core}}$. Specifically,
\beq
\Sigma_{\text{gas}} = \Sigma_{0,\text{gas}} \times \begin{cases} 
      e^{-R_{\text{core}}/H} & R \leq R_{\text{core}} \\
      e^{-R/H} & R > R_{\text{core}}\text{,}
   \end{cases}
\eeq
where $H$ is the scale length of the disk and $\Sigma_{0,\text{gas}} =
\frac{M_{\text{gas}}}{2\pi H^2}$ is the central surface density. When
$R_{\text{core}}/H=1$ and $1.4$, $\sim10\%$ and $\sim20\%$ of the gas disk
mass is removed, respectively. This mass is not returned to another component
and is thus removed from the initial conditions entirely. However, since our
assumed gas fraction is $10\%$, this only represents a total reduction in the
disk mass by a few percent. The surface density profile of the stellar disk is
unchanged, though its velocity distribution is suitably altered.

The second modification we make is to treat the initial disk particles as star
particles. In, e.g., \citet{2019MNRAS.489.4233M} the initial disk is assumed
to be a purely collisionless component with no feedback contributions.
Instead, we assume the initial disk particles are star particles so they may
contribute gas through AGB channels. This requires assigning an age and
metallicity to each star particle. We randomly draw the age and metallicity of
each star according to the age-metallicity-radius relation of
\citet{2018ApJ...865...96F}. This ignores vertical variations which is
implemented in other models \citep[e.g.][]{2020arXiv200503646S}.

\subsection{Gas Accretion}\label{ssec:gas_acc}
The method described in Section~\ref{ssec:ics} will succesfully form a bar
after a few Gyr. However, because the disk is isolated (i.e., the only gas
present in the initial conditions is in the disk), there is no source of fresh
gas to resupply the disk. We found in empirical tests that by the time the bar
had fully formed, the disk had become significantly more quiescent than what
is expected for the Milky Way, with a global SFR of <~0.5 Msun/yr and a SFR
surface density at 8 kpc about two orders of magnitude less than expected.
Such differences prohibit a faithful comparison to the Milky Way.

As a result, it is necessary to include a source of fresh gas to resupply the
galaxy. We found this to be quite problematic because several such techniques
we explored would either weaken or completely destroy the bar. We found that
it was necessary to include a somewhat \textit{ad hoc} implementation of gas
accretion from the circumgalactic medium (CGM).

Our method is inspired by the picture of a ``galactic fountain flow''
\citep{2006MNRAS.366..449F, 2008MNRAS.386..935F}. In this picture, gas is
launched from the central region of a spiral galaxy before falling back onto
the disk at a larger radii. In these models, the gas has lower angular
momentum than the disk when it rejoins the disk due to {\bf drag interactions
with the hot corona.}

We therefore have implemented a method to include ``hoses'' into AREPO. These
hoses are circular regions of radius $R_h$ placed along the $x$-axis at $x=\pm
R$. At a fixed timestep $\delta t$ each hose will spawn a number $N$ of gas
particles. The spawned gas particles have a fixed mass, a uniform initial
temperature of $10^4 \K$, and a uniform metallicity chosen to be solar. 

\subsection{Summary}\label{ssec:ics_summary}


Our goal is now to construct an initial disk galaxy with gas that, like
GALAKOS, forms a strong bar. The underlying method GALAKOS uses (namely
MakeNewDisk) can already convert a constant fraction of the initial stellar disk to a
gas disk which is initially in hydrostatic equilibrium. However, if this setup
is used na


% Standard cooling processes from hydrogen and helium undergoing two-body
% interactions, Compton cooling of cosmic microwave background photons, and
% photoionization from a uniform UV background are accounted for using tabulated
% \textsc{CLOUDY} calculations, described in detail in
% \citet{2013MNRAS.436.3031V}. In addition to these processes, metal line,
% fine-structure, and molecular cooling processes are accounted for, in order to
% allow the gas to reach $\sim 10\,\text{K}$. At high densities, self-shielding
% of the gas is accounted for using an analytic formula. Cosmic ray and
% photoelectric heating are also taken into account, important for the thermal
% stability of the cold and warm phases.

% Star formation follows a standard probabilistic approach
% \citep{2003MNRAS.339..289S}, where star particles are created and assumed to
% be a coeval population following a \citet{2001ApJ...554.1274C} initial mass
% function. Star formation efficiency ($\epsilon$) is assumed to be $0.01$. An
% additional density threshold of $100\,\text{cm}^{-3}$ and a condition on a gas
% cell to be sub-virial ($\alpha < 1$) and thus prone to gravitational collapse
% are applied.

% Finally, stellar feedback is accounted for in three different channels:
% supernovae, young massive stellar radiation, and stellar winds from OB and AGB
% stars.

Normally the next section describes the techniques the authors used.
It is frequently split into subsections, such as Section~\ref{sec:maths} below.

\subsection{Maths}
\label{sec:maths} % used for referring to this section from elsewhere

Simple mathematics can be inserted into the flow of the text e.g. $2\times3=6$
or $v=220$\,km\,s$^{-1}$, but more complicated expressions should be entered
as a numbered equation:

\begin{equation}
    x=\frac{-b\pm\sqrt{b^2-4ac}}{2a}.
	\label{eq:quadratic}
\end{equation}

Refer back to them as e.g. equation~(\ref{eq:quadratic}).

\subsection{Figures and tables}

Figures and tables should be placed at logical positions in the text. Don't
worry about the exact layout, which will be handled by the publishers.

Figures are referred to as e.g. Fig.~\ref{fig:example_figure}, and tables as
e.g. Table~\ref{tab:example_table}.

% Example figure
\begin{figure}
	% To include a figure from a file named example.*
	% Allowable file formats are eps or ps if compiling using latex
	% or pdf, png, jpg if compiling using pdflatex
	% \includegraphics[width=\columnwidth]{example}
    \caption{This is an example figure. Captions appear below each figure.
	Give enough detail for the reader to understand what they're looking at,
	but leave detailed discussion to the main body of the text.}
    \label{fig:example_figure}
\end{figure}

% Example table
\begin{table}
	\centering
	\caption{This is an example table. Captions appear above each table.
	Remember to define the quantities, symbols and units used.}
	\label{tab:example_table}
	\begin{tabular}{lccr} % four columns, alignment for each
		\hline
		A & B & C & D\\
		\hline
		1 & 2 & 3 & 4\\
		2 & 4 & 6 & 8\\
		3 & 5 & 7 & 9\\
		\hline
	\end{tabular}
\end{table}


\section{Conclusions}

The last numbered section should briefly summarise what has been done, and describe
the final conclusions which the authors draw from their work.

\section*{Acknowledgements}

The Acknowledgements section is not numbered. Here you can thank helpful
colleagues, acknowledge funding agencies, telescopes and facilities used etc.
Try to keep it short.

%%%%%%%%%%%%%%%%%%%%%%%%%%%%%%%%%%%%%%%%%%%%%%%%%%

%%%%%%%%%%%%%%%%%%%% REFERENCES %%%%%%%%%%%%%%%%%%

% The best way to enter references is to use BibTeX:

\bibliographystyle{mnras}
\bibliography{references} % if your bibtex file is called example.bib

\appendix

\section{Two-component Toomre instability criterion}
The Toomre instability criterion for a two-component fluid was first derived
by \citet{1984ApJ...276..114J}. For a mode of wavenumber $k=2\pi/\lambda$ to
be stable against gravitational collapse, the criterion for the two fluids in
an infinitesimally thin disk is that,
\beq
Q_2(k) = \left(Q_g^{-1}(k) + Q_s^{-1}(k) \right)^{-1} > 1\text{,}
\eeq
where $Q_2(k)$ is the two-component Toomre parameter and,
\beq
Q_g(k) = \frac{\kappa^2 + k^2 c_s^2}{2\pi G k \Sigma_{g,0}}\text{,}
\eeq
where $\kappa$ is the radial epicyclic frequency, $c_s$ is the sound speed of
the gas, and $\Sigma_{g,0}$ is the surface density of the gas. $Q_s(k)$ can be
obtained by replacing $c_s$ by $\sigma_R$ (the radial velocity dispersion) and
$\Sigma_{g,0}$ by $\Sigma_{s,0}$ (the stellar surface density). Note that the
familiar one-component Toomre criterion can be obtained by minimizing $Q_g(k)$
as a function of $k$.

We denote the minimum of $Q_2(k)$ as a function of $k$ by $Q_2$. We consider
only physically plausible values of $k$ between $0.06$ and $6000\,\kpc^{-1}$
(corresponding to modes of wavelength $\lambda\sim100\,\kpc$ to $1\,\pc$). The
equivalent of the one-component Toomre stability criterion is that $Q_2>1$,
such that the disk is stable against collapse of modes of all wavelengths.

\section{\mwib{} resolution levels}
We set a series of standard resolution levels similar to those in the Aquarius
convention. Due to the isolated nature of our simulations, it is convenient to
tune the particle mass to a specific number.

\begin{table}
\begin{tabular}{cccc}
resolution level & $m_{\text{DM}}$      & $m_{\text{b}}$    & $\epsilon$        \\
                 & $[\,M_{\odot}\,]$    & $[\,M_{\odot}\,]$ & $[\,\text{pc}\,]$ \\
$5$              & $2.4\times10^6$      & $4.8\times10^5$   & $80$             \\
$4$              & $3\times10^5$        & $6\times10^4$     & $40$             \\
$3$              & $3.75 \times 10^{4}$ & $7500$            & $20$             \\
$2$              & $4690$               & $938$             & $10$            \\
$1$              & $586$                & $117$             & $5$           
\end{tabular}
\end{table}

\begin{table}
\begin{tabular}{cccc}
resolution level & $m_{\text{DM}}$      & $m_{\text{b}}$    & $\epsilon$        \\
                 & $[\,M_{\odot}\,]$    & $[\,M_{\odot}\,]$ & $[\,\text{pc}\,]$ \\
$5$h             & $2.4\times10^6$      & $4.8\times10^5$   & $80$             \\
$4$h             & $3\times10^5$        & $6\times10^4$     & $40$             \\
$3$h             & $3.75 \times 10^{4}$ & $7500$            & $20$             \\
$2$h             & $4690$               & $938$             & $10$            \\
$1$h             & $586$                & $117$             & $5$           
\end{tabular}
\end{table}


%%%%%%%%%%%%%%%%%%%%%%%%%%%%%%%%%%%%%%%%%%%%%%%%%%


% Don't change these lines
\bsp	% typesetting comment
\label{lastpage}
\end{document}
