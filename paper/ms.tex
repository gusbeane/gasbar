% mnras_template.tex
%
% LaTeX template for creating an MNRAS paper
%
% v3.0 released 14 May 2015
% (version numbers match those of mnras.cls)
%
% Copyright (C) Royal Astronomical Society 2015
% Authors:
% Keith T. Smith (Royal Astronomical Society)

% Change log
%
% v3.0 May 2015
%    Renamed to match the new package name
%    Version number matches mnras.cls
%    A few minor tweaks to wording
% v1.0 September 2013
%    Beta testing only - never publicly released
%    First version: a simple (ish) template for creating an MNRAS paper

%%%%%%%%%%%%%%%%%%%%%%%%%%%%%%%%%%%%%%%%%%%%%%%%%%
% Basic setup. Most papers should leave these options alone.
\documentclass[a4paper,fleqn,usenatbib]{mnras}

% MNRAS is set in Times font. If you don't have this installed (most LaTeX
% installations will be fine) or prefer the old Computer Modern fonts, comment
% out the following line
\usepackage{newtxtext,newtxmath}
% Depending on your LaTeX fonts installation, you might get better results with one of these:
%\usepackage{mathptmx}
%\usepackage{txfonts}

% Use vector fonts, so it zooms properly in on-screen viewing software
% Don't change these lines unless you know what you are doing
\usepackage[T1]{fontenc}
\usepackage{ae,aecompl}


%%%%% AUTHORS - PLACE YOUR OWN PACKAGES HERE %%%%%

% Only include extra packages if you really need them. Common packages are:
\usepackage{graphicx}	% Including figure files
\usepackage{amsmath}	% Advanced maths commands
\usepackage{amssymb}	% Extra maths symbols

%%%%%%%%%%%%%%%%%%%%%%%%%%%%%%%%%%%%%%%%%%%%%%%%%%

%%%%% AUTHORS - PLACE YOUR OWN COMMANDS HERE %%%%%
\newcommand{\pc}{\ensuremath{\text{pc}}}
\newcommand{\kpc}{\ensuremath{\text{kpc}}}

\newcommand{\beq}{\begin{equation}}
\newcommand{\eeq}{\end{equation}}

\newcommand{\mwib}{\texttt{mwib}}

% Please keep new commands to a minimum, and use \newcommand not \def to avoid
% overwriting existing commands. Example:
%\newcommand{\pcm}{\,cm$^{-2}$}	% per cm-squared

%%%%%%%%%%%%%%%%%%%%%%%%%%%%%%%%%%%%%%%%%%%%%%%%%%

%%%%%%%%%%%%%%%%%%% TITLE PAGE %%%%%%%%%%%%%%%%%%%

% Title of the paper, and the short title which is used in the headers.
% Keep the title short and informative.
\title[Star Bar]{On Star Formation at Dynamical Resonances of the Bar}

% The list of authors, and the short list which is used in the headers.
% If you need two or more lines of authors, add an extra line using \newauthor
\author[A. Beane et al.]{
% Angus Beane,$^{1}$\thanks{E-mail: abeane@cfa.harvard.edu}
Angus~Beane$^{1}$,\thanks{E-mail: abeane@cfa.harvard.edu}
Elena~D'Onghia$^{2,3}$,
Lars~Hernquist$^{1}$,
and Federico~Marinacci$^{4}$
\\
% List of institutions
% Center for Astrophysics {\normalfont |} Harvard \& Smithsonian, 60 Garden Street, Cambridge, MA 02138, USA
$^{1}$Center for Astrophysics {\normalfont |} Harvard \& Smithsonian, 60 Garden Street, Cambridge, MA 02138, USA\\
$^{2}$Department of Astronomy, University of Wisconsin, 475 North Charter Street, Madison, WI 53706, USA\\
$^{3}$Center for Computational Astrophysics, Flatiron Institute, 162 5th Avenue, New York, NY 10010, USA\\
$^{4}$Department of Physics \& Astronomy, University of Bologna, via Gobetti 93/2, 40129 Bologna, Italy
% $^{3}$Department of Physics \& Astronomy, University of Pennsylvania, 209 South 33rd Street, Philadelphia, PA 19104, USA
}

% These dates will be filled out by the publisher
\date{Accepted XXX. Received YYY; in original form ZZZ}

% Enter the current year, for the copyright statements etc.
\pubyear{2019}

% Don't change these lines
\begin{document}
\label{firstpage}
\pagerange{\pageref{firstpage}--\pageref{lastpage}}
\maketitle

% Abstract of the paper
\begin{abstract}
Stars are known to be captured at the dynamical resonances of the Milky Way
bar, forming moving groups that are seen in the solar neighborhood. In
principle, gas can also be captured at these resonances which, if allowed to
cool, might trigger star formation -- predicting an enhanced number of
star-forming regions at these resonances. However, the efficiency of stellar
and radiation feedback may suppress the star formation process. 
\end{abstract}

% Select between one and six entries from the list of approved keywords.
% Don't make up new ones.
\begin{keywords}
Galaxy: disc -- Galaxy: kinematics and dynamics -- stars: kinematics and dynamics
\end{keywords}

%%%%%%%%%%%%%%%%%%%%%%%%%%%%%%%%%%%%%%%%%%%%%%%%%%

%%%%%%%%%%%%%%%%% BODY OF PAPER %%%%%%%%%%%%%%%%%%

\section{Introduction}

This is a simple template for authors to write new MNRAS papers.
See \texttt{mnras\_sample.tex} for a more complex example, and \texttt{mnras\_guide.tex}
for a full user guide.

All papers should start with an Introduction section, which sets the work
in context, cites relevant earlier studies in the field by \citet{Others2013},
and describes the problem the authors aim to solve \citep[e.g.][]{Author2012}.

\section{Methods, Observations, Simulations etc.}

Normally the next section describes the techniques the authors used.
It is frequently split into subsections, such as Section~\ref{sec:maths} below.

\subsection{Maths}
\label{sec:maths} % used for referring to this section from elsewhere

Simple mathematics can be inserted into the flow of the text e.g. $2\times3=6$
or $v=220$\,km\,s$^{-1}$, but more complicated expressions should be entered
as a numbered equation:

\begin{equation}
    x=\frac{-b\pm\sqrt{b^2-4ac}}{2a}.
	\label{eq:quadratic}
\end{equation}

Refer back to them as e.g. equation~(\ref{eq:quadratic}).

\subsection{Figures and tables}

Figures and tables should be placed at logical positions in the text. Don't
worry about the exact layout, which will be handled by the publishers.

Figures are referred to as e.g. Fig.~\ref{fig:example_figure}, and tables as
e.g. Table~\ref{tab:example_table}.

% Example figure
\begin{figure}
	% To include a figure from a file named example.*
	% Allowable file formats are eps or ps if compiling using latex
	% or pdf, png, jpg if compiling using pdflatex
	% \includegraphics[width=\columnwidth]{example}
    \caption{This is an example figure. Captions appear below each figure.
	Give enough detail for the reader to understand what they're looking at,
	but leave detailed discussion to the main body of the text.}
    \label{fig:example_figure}
\end{figure}

% Example table
\begin{table}
	\centering
	\caption{This is an example table. Captions appear above each table.
	Remember to define the quantities, symbols and units used.}
	\label{tab:example_table}
	\begin{tabular}{lccr} % four columns, alignment for each
		\hline
		A & B & C & D\\
		\hline
		1 & 2 & 3 & 4\\
		2 & 4 & 6 & 8\\
		3 & 5 & 7 & 9\\
		\hline
	\end{tabular}
\end{table}


\section{Conclusions}

The last numbered section should briefly summarise what has been done, and describe
the final conclusions which the authors draw from their work.

\section*{Acknowledgements}

The Acknowledgements section is not numbered. Here you can thank helpful
colleagues, acknowledge funding agencies, telescopes and facilities used etc.
Try to keep it short.

%%%%%%%%%%%%%%%%%%%%%%%%%%%%%%%%%%%%%%%%%%%%%%%%%%

%%%%%%%%%%%%%%%%%%%% REFERENCES %%%%%%%%%%%%%%%%%%

% The best way to enter references is to use BibTeX:

\bibliographystyle{mnras}
\bibliography{references} % if your bibtex file is called example.bib

\appendix

\section{Two-component Toomre instability criterion}
The Toomre instability criterion for a two-component fluid was first derived
by \citet{1984ApJ...276..114J}. For a mode of wavenumber $k=2\pi/\lambda$ to
be stable against gravitational collapse, the criterion for the two fluids in
an infinitesimally thin disk is that,
\beq
Q_2(k) = \left(Q_g^{-1}(k) + Q_s^{-1}(k) \right)^{-1} > 1\text{,}
\eeq
where $Q_2(k)$ is the two-component Toomre parameter and,
\beq
Q_g(k) = \frac{\kappa^2 + k^2 c_s^2}{2\pi G k \Sigma_{g,0}}\text{,}
\eeq
where $\kappa$ is the radial epicyclic frequency, $c_s$ is the sound speed of
the gas, and $\Sigma_{g,0}$ is the surface density of the gas. $Q_s(k)$ can be
obtained by replacing $c_s$ by $\sigma_R$ (the radial velocity dispersion) and
$\Sigma_{g,0}$ by $\Sigma_{s,0}$ (the stellar surface density). Note that the
familiar one-component Toomre criterion can be obtained by minimizing $Q_g(k)$
as a function of $k$.

We denote the minimum of $Q_2(k)$ as a function of $k$ by $Q_2$. We consider
only physically plausible values of $k$ between $0.06$ and $6000\,\kpc^{-1}$
(corresponding to modes of wavelength $\lambda\sim100\,\kpc$ to $1\,\pc$). The
equivalent of the one-component Toomre stability criterion is that $Q_2>1$,
such that the disk is stable against collapse of modes of all wavelengths.

\section{\mwib{} resolution levels}
We set a series of standard resolution levels similar to those in the Aquarius
convention. Due to the isolated nature of our simulations, it is convenient to
tune the particle mass to a specific number.

\begin{table}
\begin{tabular}{cccc}
resolution level & $m_{\text{DM}}$      & $m_{\text{b}}$    & $\epsilon$        \\
                 & $[\,M_{\odot}\,]$    & $[\,M_{\odot}\,]$ & $[\,\text{pc}\,]$ \\
$5$              & $2.4\times10^6$      & $4.8\times10^5$   & $80$             \\
$4$              & $3\times10^5$        & $6\times10^4$     & $40$             \\
$3$              & $3.75 \times 10^{4}$ & $7500$            & $20$             \\
$2$              & $4690$               & $938$             & $10$            \\
$1$              & $586$                & $117$             & $5$           
\end{tabular}
\end{table}

\begin{table}
\begin{tabular}{cccc}
resolution level & $m_{\text{DM}}$      & $m_{\text{b}}$    & $\epsilon$        \\
                 & $[\,M_{\odot}\,]$    & $[\,M_{\odot}\,]$ & $[\,\text{pc}\,]$ \\
$5$h             & $2.4\times10^6$      & $4.8\times10^5$   & $80$             \\
$4$h             & $3\times10^5$        & $6\times10^4$     & $40$             \\
$3$h             & $3.75 \times 10^{4}$ & $7500$            & $20$             \\
$2$h             & $4690$               & $938$             & $10$            \\
$1$h             & $586$                & $117$             & $5$           
\end{tabular}
\end{table}


%%%%%%%%%%%%%%%%%%%%%%%%%%%%%%%%%%%%%%%%%%%%%%%%%%


% Don't change these lines
\bsp	% typesetting comment
\label{lastpage}
\end{document}
